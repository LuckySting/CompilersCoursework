\begin{titlepage}
    \begin{minipage}{0.85\textwidth}
        \raggedleft
        \begin{center}
            \fontsize{11pt}{0.3\baselineskip}\selectfont \textbf{Министерство науки и высшего образования Российской Федерации \\ Федеральное государственное бюджетное образовательное учреждение \\ высшего образования \\ <<Московский государственный технический университет \\ имени Н.Э. Баумана \\ (национальный исследовательский университет)>> \\ (МГТУ им. Н.Э. Баумана)}
        \end{center}
    \end{minipage}

    \begin{center}
        \fontsize{11pt}{0.1\baselineskip}\selectfont
        \noindent\makebox[\linewidth]{\rule{\textwidth}{4pt}} \makebox[\linewidth]{\rule{\textwidth}{1pt}}
    \end{center}

    \begin{flushright}
        \begin{minipage}{8cm}
            \fontsize{11pt}{0.8\baselineskip}\selectfont
            \begin{signstabular}[0.4]{p{4.1cm} >{\centering\arraybackslash}p{2.5cm}}
                УТВЕРЖДАЮ & \\
                \\
                Заведующий кафедрой & \uline{\hfill ИУ7 \hfill} \\
                & \scriptsize (Индекс) \\
                \\
                \uline{\mbox{\hspace*{4.1cm}}} & \uline{\hfill И. В. Рудаков \hfill} \\
                & \scriptsize (И. О. Фамилия) \\
                \\
                \multicolumn{2}{l}{
                    <<\uline{\mbox{\hspace*{1cm}}}>>
                    \uline{\mbox{\hspace*{4cm}}}
                    20\uline{\mbox{\hspace*{0.8cm}}}г.
                }
            \end{signstabular}
        \end{minipage}
    \end{flushright}

    \vfill

    \begin{center}
        \fontsize{16pt}{0.6\baselineskip}\selectfont

        \textbf{ЗАДАНИЕ \\ на выполнение курсового проекта}
    \end{center}

    \begin{flushleft}
        \fontsize{11pt}{0.4cm}\selectfont
        по дисциплине \uline{\hfill Конструирование компиляторов \hfill}

        Студент группы \uline{\makebox[3cm]{ИУ7-22М}}

        \begin{signstabular}[0.5]{@{}c}
            \uline{\makebox[\textwidth]{Третьяк Роман Дмитриевич}} \\
            \scriptsize (Фамилия, имя, отчество)
        \end{signstabular}

        Тема курсового проекта \uline{\hfill Компилятор подмножества языка программирования Golang \hfill}

        Направленность КП (учебный, исследовательский, практический, производственный, др.)

        \uline{\hfill Учебный \hfill}

        Источник тематики (кафедра, предприятие, НИР) \uline{\hfill кафедра \hfill}

        \textbf{Задание:} \uline{\hfill Разработать компилятор подмножества языка программирования Golang,
            используя генератор парсеров ANTLR4 и инструмент кодогенерации LLVM.  \hfill}

        \textbf{Оформление курсового проекта:}

        Расчетно-пояснительная записка на \uline{ 20-30 } листах формата А4.

        \uline{\hfill Расчетно-пояснительная записка должна содержать постановку, введение, полное описание всех стадий
        компиляции заключение, список использованной литературы  \hfill}

        Перечень графического (иллюстративного) материала (чертежи, плакаты, слайды и т.п.)

        Дата выдачи задания <<\uline{\mbox{\hspace*{1cm}}}>>\uline{\mbox{\hspace*{4cm}}}20\uline{\mbox{\hspace*{0.8cm}}}г.

    \end{flushleft}

    \begin{flushleft}
        \fontsize{11pt}{0.6\baselineskip}\selectfont

        \begin{signstabular}[0.5]{@{}p{\textwidth - 8.9cm} >{\centering\arraybackslash}p{4cm} >{\centering\arraybackslash}p{4cm}}
            Руководитель курсового проекта & \uline{\mbox{\hspace*{4cm}}} & \uline{\hfill \textbf{\teacher} \hfill} \\
            & \scriptsize (Подпись, дата) & \scriptsize (И. О. Фамилия)
        \end{signstabular}

        \vspace{1cm}

        \begin{signstabular}[0.5]{@{}p{\textwidth - 8.9cm} >{\centering\arraybackslash}p{4cm} >{\centering\arraybackslash}p{4cm}}
            Студент & \uline{\mbox{\hspace*{4cm}}} & \uline{\hfill \textbf{Р. Д. Третьяк} \hfill} \\
            & \scriptsize (Подпись, дата) & \scriptsize (И. О. Фамилия)
        \end{signstabular}

        \label{tab:todo}
    \end{flushleft}

    \begin{flushleft}
        \fontsize{11pt}{0.4cm}\selectfont
        \uline{Примечание:} Задание оформляется в двух экземплярах: один выдается студенту, второй хранится на
        кафедре.
    \end{flushleft}
    \vfill
\end{titlepage}