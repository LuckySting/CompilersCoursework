%----------------------- Преамбула -----------------------
\documentclass[ut8x, 14pt, oneside, a4paper]{extarticle}

\usepackage{extsizes} % Для добавления в параметры класса документа 14pt
%
% Для работы с несколькими языками и шрифтом Times New Roman по-умолчанию
\usepackage[russian]{babel}
\usepackage{fontspec}
\setmainfont{Times New Roman}
%\usepackage{times}ot
%
% ГОСТовские настройки для полей и абзацев
\usepackage{float}
\usepackage[left=30mm,right=10mm,top=20mm,bottom=20mm]{geometry}
\usepackage{misccorr}
\usepackage{indentfirst}
\usepackage{enumitem}
\setlength{\parindent}{1.25cm}
%\linespread{1.3}
\renewcommand{\baselinestretch}{1.5}
\setlist{nolistsep} % Отсутствие отступов между элементами \enumerate и \itemize

% Дополнительное окружения для подписей
\usepackage{array}
\newenvironment{signstabular}[1][1]{
    \renewcommand*{\arraystretch}{#1}
    \tabular
    }{
    \endtabular
}

% Переопределение стандартных \section, \subsection, \subsubsection по ГОСТу;
% Переопределение их отступов до и после для 1.5 интервала во всем документе
\usepackage{titlesec}

\titleformat{\section}[block]
{\bfseries\normalsize\raggedright}{\thesection}{1em}{}

\titleformat{\subsection}[hang]
{\bfseries\normalsize\raggedright}{\thesubsection}{1em}{}
\titlespacing\subsection{\parindent}{\parskip}{\parskip}

\titleformat{\subsubsection}[hang]
{\bfseries\normalsize\raggedright}{\thesubsubsection}{1em}{}
\titlespacing\subsubsection{\parindent}{\parskip}{\parskip}

% Работа с изображениями и таблицами; переопределение названий по ГОСТу
\usepackage{caption}
\captionsetup[figure]{name={Рисунок},labelsep=endash}
\captionsetup[table]{singlelinecheck=false, labelsep=endash}

\usepackage{xcolor,colortbl}
\usepackage{mathtools}
\usepackage{graphicx}
%\usepackage{slashbox} % Диагональное разделение первой ячейки в таблицах

% Цвета для гиперссылок и листингов
\usepackage{color}

% Настройки оглавления
\usepackage{tocloft}
\setlength{\cftpartindent}{0em}
\setlength{\cftchapindent}{0em}
\setlength{\cftsecindent}{0em}
\setlength{\cftsubsecindent}{0em}
\setlength{\cftsubsubsecindent}{0em}
\setlength{\cftbeforesecskip}{0}
\renewcommand{\cftsecleader}{\cftdotfill{\cftXdotsep}}
\renewcommand{\cftdotsep}{0}
\renewcommand{\cftsecaftersnum}{.}
\renewcommand{\cftsecfont}{\normalfont}
\renewcommand{\cftsecpagefont}{\normalfont}
\cftsetpnumwidth{1.2em}
\usepackage{hyperref}

\hypersetup{
    linktoc=all,
    linkcolor=black,
    colorlinks=true,
}

% Листинги
%\setsansfont{Arial}
\setmonofont{Courier New}
% Цвета для гиперссылок и листингов
\definecolor{comment}{rgb}{0,0.5,0}
\definecolor{plain}{rgb}{0.2,0.2,0.2}
\definecolor{string}{rgb}{0.91,0.45,0.32}
\hypersetup{citecolor=black}

\usepackage{listings}
\lstset{
    basicstyle=\footnotesize\ttfamily,
    language=Python, % Или другой ваш язык -- см. документацию пакета
    commentstyle=\color{comment},
    numbers=left,
    numberstyle=\tiny\color{plain},
    numbersep=5pt,
    tabsize=4,
    extendedchars=\true,
    breaklines=true,
    keywordstyle=\color{blue},
    frame=b,
    stringstyle=\ttfamily\color{string}\ttfamily,
    showspaces=false,
    showtabs=false,
    xleftmargin=17pt,
    framexleftmargin=17pt,
    framexrightmargin=5pt,
    framexbottommargin=4pt,
    showstringspaces=false,
    inputencoding=utf8x,
    keepspaces=true
}

\DeclareCaptionLabelSeparator{line}{\ --\ }
\DeclareCaptionFont{white}{\color{white}}
\DeclareCaptionFormat{listing}{\colorbox[cmyk]{0.43,0.35,0.35,0.01}{\parbox{\textwidth}{\hspace{15pt}#1#2#3}}}
\captionsetup[lstlisting]{
    format=listing,
    labelfont=white,
    textfont=white,
    singlelinecheck=false,
    margin=0pt,
    font={bf,footnotesize},
    labelsep=line
}

\usepackage{ulem} % Нормальное нижнее подчеркивание
\usepackage{hhline} % Двойная горизонтальная линия в таблицах
\usepackage[figure,table]{totalcount} % Подсчет изображений, таблиц
\usepackage{rotating} % Поворот изображения вместе с названием
\usepackage{lastpage} % Для подсчета числа страниц

\bibliographystyle{ugost2008}

\usepackage{amsmath}
\usepackage{amssymb}
%%\usepackage{slashbox}