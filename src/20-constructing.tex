\section{Конструкторский раздел}\label{sec:construct}

\subsection{Структура компилятора}\label{subsec:structure}
Компилятор состоит из 3-ех модулей:
\begin{itemize}
    \item лексический анализатор, преобразовывающий текст программы в поток токенов;
    \item синтаксический анализатор, строящий AST-дерево;
    \item генератор LLRM IR кода.
\end{itemize}

\subsection{Генерация лексического и синтаксического анализатора}\label{subsec:lex_synt_gen}
Для создания анализатора в работе используется ANTLR4~\cite{antlr}.
В качестве входных данных для него выступает файл с описанием грамматики исходного языка.
Данный файл содержит только правила грамматики без добавления кода, исполнение которого соответствует применению определённых правил.
Подобное разделение позволяет использовать один и тот же файл грамматики для построения различных приложений
(например, компиляторов, генерирующих код для различных сред исполнения).

На основе правил заданной грамматики языка ANTLR генерирует класс нисходящего рекурсивного синтаксического анализатора.
Для каждого правила грамматики в полученном классе имеется свой рекурсивный метод.
Разбор входной последовательности начинается с корня синтаксического дерева и заканчивается в листьях.

Сгенерированный ANTLR синтаксический анализатор выдаёт абстрактное синтаксическое дерево в чистом виде, и реализует методы для
его построения и последующего обхода.
Дерево разбора для заданной входной последовательности символов можно получить, вызвав метод, соответствующий аксиоме
в исходной грамматике языка.
В грамматике языка Golang аксиомой является нетерминал sourceFile, поэтому построение дерева следует начинать с вызова
метода sourceFile() объекта класса синтаксического анализатора, являющегося корнем дерева.

\subsection{Обнаружение и обработка лексических и синтаксических ошибок}\label{subsec:errors}
Все ошибки, которые обнаруживаются лексическим и синтаксическим анализаторами ANTLR, по умолчанию выводятся в стандартный поток вывода ошибок.
Данные ошибки возможно перехватить стандартным обработчиком ошибок языка на котором ведется разработка компилятора.

\subsection{Генерация кода}\label{subsec:codegen}
Помимо парсера ANTLR также генерирует класс, реализующий обход полученного дерева, для каждого типа узлов дерева
в классе присутсвует метод его обработки, называние метода начинается со слова "visit". В листинге~\ref{lst:visit_var_spec}
представлен пример метода обработки узла операции объявления функции.
В методе происходит генерация инструкции выделения памяти, а также создание записи в таблице переменных текущей области видимости.
Кроме того производится попытка привидения типа значения к типу переменной, в случае, если помимо объявления переменной
происходит также присвоение ей значения.

\begin{lstlisting}[language=Python,caption={Метод обработки объявления/определения переменной},label={lst:visit_var_spec}]
def visitVarSpec(self, ctx: GoParser.VarSpecContext):
    variable_name = ctx.identifierList().getText()
    variable_type = self.visitType_(ctx.type_())
    variable = self.current_builder.alloca(variable_type, name=variable_name)
    self.variables_scopes[self.current_scope_name][variable_name] = variable
    if ctx.expressionList():
        value = self.visitExpressionList(ctx.expressionList())
        value = self.maybe_convert_type(value, variable.type.pointee)
        self.current_builder.store(value, variable)
    return variable
\end{lstlisting}

Отедльного внимания заслуживают вспомогательные методы согласования типов. Так как LLVMIR требует строго соответсвия
типов аргументов и операции, зачастую необходимо перед выполнением инструкции операции выполнить приведение типов,
загрузку значения из памяти. Для этих целей служат вспомогательные функции, одна из которых приведена в листинге~\ref{lst:maybe_convert_type}

\begin{lstlisting}[language=Python,caption={Вспомогательный метод согласования типов},label={lst:helpers}]
def maybe_convert_type(self, value, target_type):
    if isinstance(value, ir.Constant):
        value.type = target_type
        return value
    if isinstance(value, ir.AllocaInstr):
        value = self.current_builder.load(value, name=value.name)
    if value.type == target_type:
        return value
    if target_type.intrinsic_name == 'f64':
        value = self.current_builder.sitofp(value, ir.DoubleType())
        return value
    if target_type.intrinsic_name == 'i1':
        value = self.current_builder.icmp_signed('>', value, value.type(0))
        return value
    if target_type.intrinsic_name in CONVERTABLE_INT_EXT[value.type.intrinsic_name]:
        value = self.current_builder.sext(value, target_type)
        return value
    if target_type.intrinsic_name in CONVERTABLE_INT_TRUNC[value.type.intrinsic_name]:
        value = self.current_builder.trunc(value, target_type)
        return value
    raise Exception('Filed to convert type to target')
\end{lstlisting}

Так как методы разбора узлов вызывают соответствующие методы разбора дочерних узлов, то обработка всего дерева
заключается в разборе корневого элемента вызовом метода "visitSourceFile".
Когда все узлы дерева обработаны, полученаня последовательность инструкций на языке LLVM IR верифицируется и записывается
в файл.
