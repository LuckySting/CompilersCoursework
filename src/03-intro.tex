\phantomsection
\section*{Введение}
\addcontentsline{toc}{section}{Введение}

Подавляющая часть программных продуктов в текущих реалиях разрабатваются
с применением языков высокого уровня.
Они позволяют программисту, оперируя абстракциями языка, сконцентрироваться на логике создаваемой
программы, абстрагируясь от низкоуровневых особенностей и рутинных задач.
Однако, программа на высокуровневом языке программирования не может в чистом
виде быть выполненной компьютером, требуется перевод в код, воспринимаемый
платфорой-исполнителем. Эту задачу решает компилятор.

Компилятор -- комплекс программных средств, принимающий на вход программный код,
написанный на высокоуровневом языке программаирования. Компилятор производит
анализ полученного исходного кода на предмет ошибок, в некоторых случаях производит
оптимизацию программного кода, и генерирует эквивалентный код на целевом языке.
Целевым языком чаще всего является платфомозависимый ассемблерный код или код,
воспринимаемый некоторой виртуальной машиной.

Целью работы является разработка компилятора подмножества языка Golang для
виртуальной машины LLVM.