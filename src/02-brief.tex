\phantomsection
\section*{РЕФЕРАТ}
\addcontentsline{toc}{section}{РЕФЕРАТ}

Расчетно-пояснительная записка~\pageref{LastPage} с.,
\totalfigures\ рис., 4 источник., 3 прил.

Цель работы – разработать компилятор подмножества язка программирования
Golang для платформы LLVM.

Задачи работы:
\begin{itemize}
    \item проанализировать способы и алгоритмы построения компиляторов;
    \item выбрать инструменты для разработки компилятора;
    \item спроктировать компилятор;
    \item реализовать компилятор на языке Python.
\end{itemize}

В первой части работы проанализированы алгоритмы и инструменты,
используемые при разработке компиляторов, описаны особенности языка
Golang и платформы LLVM. Во второй части описаны способы реализации
составляющих компилятора. В третьей части описаны сборка и использование
разработанного компилятора.